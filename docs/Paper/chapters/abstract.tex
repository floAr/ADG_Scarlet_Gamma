\chapter*{Inhaltsangabe}

In vielen modernen PC-Rollenspielen tritt die Schwierigkeit auf dem Spieler Handlungsfreiraum einzuräumen. Es stellt sich die Frage, ob eine Übertragung der Freiheit aus Pen-\&-Paper-Rollenspielen auf digitale Rollenspiele möglich ist.

Unterschiedliche Ansätze befassen sich mit einer dnyamischen Anpassung der Handlung. Wir beschäftigen uns hingegen mit der Frage, ob der Computer überhaupt als Medium in Frage kommt, wenn Handlungsfreiheit gefordert ist. Dazu übertragen wir das Pen-\&-Paper-Spiel direkt und fordern, dass einer der Spieler die Sonderrolle des allmächtigen Spielleiters übernimmt. Das von uns entwickelte Konzept versucht möglichst viel Komfort zu bieten und gleichzeitig die Spieler nicht in ihrem Handlungsfreiraum einzuschränken. Dazu haben wir eine komponentenbasierte Welt in ein Multiplayer-Spiel eingebettet und in Nutzertests evaluiert. Das Ziel der Handlungsfreiheit wurde erreicht und es konnte gezeigt werden, dass der Computer gleichzeitig neue Komfortfunktionen bieten kann, welche dem analogen Vorgänger fehlen.