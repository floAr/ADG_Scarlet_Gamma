\chapter*{Inhaltsangabe}

Bei vielen modernen PC-Rollenspielen tritt die Schwierigkeit auf dem Spieler Handlungsfreiraum einzuräumen. Es stellt sich die Frage, ob es nicht möglich ist die Freiheit, die es im Vorgänger, den Pen-\&-Paper-Spielen, gab zu übertragen. Es gibt unterschiedlich Ansätze in die Richtung die Handlung dynamisch anzupassen.

Wir beschäftigen uns hingegen mit der Frage, ob der Computer überhaupt als Medium in Frage kommt. Dazu übertragen wir das Pen-\&-Paper-Spiel direkt und fordern, dass einer der Spieler die Sonderrolle des allmächtigen Spielleiters übernimmt. Das von uns entwickelte Konzept versucht weitestgehend viel Komfort zu bieten und gleichzeitig die Spieler nicht in ihrem Handlungsfreiraum einzuschränken. Wir haben dazu eine Komponenten basierte Welt in ein Multiplayer-Spiel eingebettet und in Nutzertests evaluiert. Das Ziel der Handlungsfreiheit wurde erreicht und es hat sich gezeigt, dass der Computer gleichzeitig neue Komfortfunktionen bieten kann, welche dem Vorgänger auf Papier fehlen.