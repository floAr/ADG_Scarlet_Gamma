\chapter{Grundlagen}
\label{background}

%Alle Quellen hier referenzieren!

%- Allgemeine Wissensgrundlagen des Fachgebiets
%- Spezielle Grundlagen, die für das Verständnis erforderlich sind
%- Rahmenbedingungen für die Arbeit
%- Ausführungen zum Stand des Wissens / der Technik
%Als Leitprinzip gilt: Nur Informationen erwähnen, die
%- später benötigt werden,
%- notwendig sind, um die Arbeit oder ihre Motivation zu verstehen
%Das heißt insbesondere,
%- keine Inhalte aus Lehrbüchern, außer
%- diese werden benötigt, um Problemstellung oder Lösungsweg zu definieren.

%Was sind Pen&Paper spiele?
%Was tun die Spieler?
%Was tut der Spielleiter?


\section{Pen\&Paper Spiele}
\label{sec:PenPaperSpiele}

%\begin{itemize}
%	\item Kurze Erklärung wie so etwas abläuft
%	\item Übersicht über bekannte PnP Spiele, was ist an D\&D besonders?
%	\item \textit{[Abgrenzung zu normalen RPGs]} NEIN, da in \ref{sec:DigitaleRollenspiele}
%\end{itemize}

Pen\&Paper Rollenspiele (im folgenden auch als PnP bezeichnet) stellen eine Urform des Rollenspiels dar.\todo{quelle} Mehrere Spieler treffen sich um gemeinsam ein Abenteuer zu erspielen. Einer der Spieler nimmt dabei die Rolle des Spielleiters ein. Die Spieler werden von dem Spielleiter durch ein fiktives Abenteuer geführt, wobei in der Regel der genaue Verlauf der Geschichte stark von der Interaktion der Mitspieler abhängt.\cite{Apperley2006}\newline 
Eine Besonderheit von PnP Rollenspielen ist, dass keinerlei reales Rollenspiel stattfindet. Zwar erhalten die Spieler Karten und begleitende Materialien von dem Spielleiter, das eigentliche Spiel findet jedoch in der Phantasie der Spieler statt \cite{Copier2005}. Die Interaktion mit der Spielwelt erfolgt verbal, Spieler treffen Aussagen über Aktionen die der Charakter ausführen soll. Der Spielleiter validiert diese Aussagen und antwortet mit dem Resultat der Aktion. Durch diese Art der Kommunikation ergibt sich oftmals ein dynamischen Abenteuer, welches speziell auf die Spielergruppe zugeschnitten ist. \cite{Drachen2008}

\subsection{Spielleiter}
\label{sec:Spielleiter}
%Beschreibung der Tätigkeiten und Funktion von Spielleitern beim PnP\newline
%Abdecken von:
%\begin{itemize}
%	\item Konzeption und Umsetzung eines Abenteuers ('offline' vor der Spielrunde)
%	\item Durchführen des Spieles ('online' währen der Runde mit dem Spielern zusammen)
%\end{itemize}
Der Spielleiter hat bei einem PnP Rollenspiel die Aufgabe eine Geschichte für die Spielrunde zu erdenken und die Teilnehmer während des Verlaufes anzuleiten. Im Vorfeld der Spielrunde konzipiert der Spielleiter dafür eine Ausarbeitung eines Abenteuers. Meist werden hierfür Kerncharaktere und wichtige Entscheidungspunkte / Szenen ausgearbeitet, während sich der genaue Verlauf des Abenteuers beim Spielen ergibt. \ref{fig:storyflow_pnp} zeigt, wie auf mehreren Ideen des Spielleiters durch Einflussnahme und Interaktion der Spieler eine Geschichte wird. Um diese Entwicklung zu ermöglichen muss der Spielleiter verschiedene Funktionen erfüllen.\newline
\cite{Arinbjarnar} nennt drei verschiedene Kernaufgaben des Spielleiters: \emph{Storyteller}, \emph{Actor} und \emph{Judge}. Zum einen ist es die Aufgabe des Spielleiters die Geschichte aus objektiver Sicht zu erzählen und die Handlung mit den Spielern voranzutreiben. Gleichzeitig nimmt er jedoch auch die Rolle und den Standpunkt aller NSCs ein und vermittelt zwischen ihnen und den Spielern. Als letzte Aufgabe achtet der Spielleiter auf die Einhaltung aller Regeln und trifft Entscheidungen falls es zu Unstimmigkeiten kommt.
\begin{figure}
	\centering
		\includegraphics[width=1.00\textwidth]{media/storyflow_pnp.png}
	\caption{Entwicklung einer Geschichte nach \cite{Tychsen2006a}}
	\label{fig:storyflow_pnp}
\end{figure}




\subsection{Spielermotivation}
\label{sec:Spielermotivation}

\begin{itemize}
	\item Hervorheben was die PnP Spieler wollen (nach Möglichkeit belegen: Paper, Foren, unsere Testspieler, Pathfinder Entwickler)
	
	\begin{itemize}
		\item Kreative Interaktive Geschichte
		\item Soziale Komponente
		\item 'Gemeinsam Abenteuer erleben'
	\end{itemize}
\end{itemize}
Ein wichtiger Punkt bei der Beantwortung der wissenschaftlichen Frage ist die treibenden Motivation der Spieler. Nur ein Verständnis der Wünsche und Ansprüche von Spielern an ein PnP-RPG erlaubt es dieses Gefühl in einem digitalen Spiel zu erzeugen. 

\section{Digitale Rollenspiele}
\label{sec:DigitaleRollenspiele}
%
%Beschreibung eines Standard Pc-Rollenspiels mit Mechaniken und Möglichkeiten\newline
%Dient später der Abgrenzung zu unserem Prototypen
%\begin{itemize}
%	\item DSA Reihe
%	\item D\&D Online
%	\item Neverwinter Nights
%	\item Elder Scrolls
%	\item Baldurs Gate
%	\item Icewind Dale
%\end{itemize}
%Besonders auf die D\&D basierten eingehen, den rest nur als Beispiele für andere Systeme.

Digitale Rollenspiele basieren häufig auf einem ähnlichen Regelwerk wie Pen\&Paper Spiele. Die exakten Formeln werden in ein festes Regelwerk überführt, welches der Computer zur Evaluierung aller Werte und Möglichkeiten nutzen kann. Anders als bei PnP Spielen wir den digitalen Spielen jedoch ein starker Fokus auf die visuelle Repräsentation gelegt \cite{Tychsen2006}. Diese visuelle Repräsentation ersetzt die imaginäre Spielwelt von PnP Rollenspielen. Dadurch ist keine kontinuierliche Kommunikation und Synchronisation von Ereignissen und Umgebungen zwischen Spielleiter und Spielern mehr nötig \cite{Drachen2008}. \newline
Gleichzeitig geht ein Großteil der Kontrolle über den Spielverlauf von dem Spieler and den Computer über. Der Spieler muss, anders als bei PnP Spielen, nicht zwingend Verständnis über die Regeln des Spieles haben. Alle Entscheidungen werden im Hintergrund durch den Computer getroffen und in der visuellen Repräsentation reflektiert. Dieses strikte Vorgehen verhindert das variieren von Regeln, während ein PnP Spielleiter kreative Entscheidungen auf Basis der bekannten Regeln treffen kann, ist der Computer an exakt den vordefinierten Regelsatz gebunden. \cite{Drachen2008}

Als Beispiel sei das Rollenspiel \emph{Neverwinter Nights} genannt, welches auf dem D\&D Regelwerk der 3. Edition basiert. Mit der AURORA Engine ist möglich ein eigenes Abenteuer zu erstellen und mit anderen Spielern über das Netzwerk zu bespielen. Sogar rudimentäre Spielleiter Tools sind vorhanden. Allerdings kann kein Spieler jemals das Regelsystem des Spieles verlassen und eine  Aktion ausführen, welche nicht von den Entwicklern implementiert wurde.



\section{Bestehende Ansätze?}
\label{sec:BekannteAnsaetze}
Paper und Programme die unserem Thema ähneln, jeweils mit Abgrenzung warum sie die Frage nicht zufriedenstellend (oder wenigstens schlechter als wir) beantworten können
\begin{itemize}
	\item RPG Maker
	\item NWN
	\item GM-AI
	\item Tools die ähnliches leisten
\end{itemize}

Das in \ref{sec:DigitaleRollenspiele} genannte \emph{Neverwinter Nights} wird mit einem Editor und eine Spielleiter Tool ausgeliefert, welche das Erstellen und leiten eigener Abenteuer erlauben. Hierbei werden dem Spielleiter erweiterte Möglichkeiten in der Spielwelt zugesprochen, allerdings ist der Spielleiter dabei stets an die im Spiel implementierten Regeln gebunden. \cite{Tychsen2006a}