\chapter{Grundlagen}
\label{background}

Alle Quellen hier referenzieren!

%- Allgemeine Wissensgrundlagen des Fachgebiets
%- Spezielle Grundlagen, die für das Verständnis erforderlich sind
%- Rahmenbedingungen für die Arbeit
%- Ausführungen zum Stand des Wissens / der Technik
%Als Leitprinzip gilt: Nur Informationen erwähnen, die
%- später benötigt werden,
%- notwendig sind, um die Arbeit oder ihre Motivation zu verstehen
%Das heißt insbesondere,
%- keine Inhalte aus Lehrbüchern, außer
%- diese werden benötigt, um Problemstellung oder Lösungsweg zu definieren.

%Was sind Pen&Paper spiele?
%Was tun die Spieler?
%Was tut der Spielleiter?


\section{Pen\&Paper Spiele}
\label{sec:PenPaperSpiele}

\begin{itemize}
	\item Kurze Erklärung wie so etwas abläuft
	\item Übersicht über bekannte PnP Spiele, was ist an D\&D besonders?
	\item \textit{[Abgrenzung zu normalen RPGs]} NEIN, da in \ref{sec:DigitaleRollenspiele}
\end{itemize}


\subsection{Spielleiter}
\label{sec:Spielleiter}
Beschreibung der Tätigkeiten und Funktion von Spielleitern beim PnP\newline
Abdecken von:
\begin{itemize}
	\item Konzeption und Umsetzung eines Abenteuers ('offline' vor der Spielrunde)
	\item Durchführen des Spieles ('online' währen der Runde mit dem Spielern zusammen)
\end{itemize}


\subsection{Spielermotivation}
\label{sec:Spielermotivation}

\begin{itemize}
	\item Hervorheben was die PnP Spieler wollen (nach Möglichkeit belegen: Paper, Foren, unsere Testspieler, Pathfinder Entwickler)
	
	\begin{itemize}
		\item Kreative Interaktive Geschichte
		\item Soziale Komponente
		\item 'Gemeinsam Abenteuer erleben'
	\end{itemize}
\end{itemize}


\section{Digitale Rollenspiele}
\label{sec:DigitaleRollenspiele}

Beschreibung eines Standard Pc-Rollenspiels mit Mechaniken und Möglichkeiten\newline
Dient später der Abgrenzung zu unserem Prototypen
\begin{itemize}
	\item DSA Reihe
	\item D\&D Online
	\item Neverwinter Nights
	\item Elder Scrolls
	\item Baldurs Gate
	\item Icewind Dale
\end{itemize}
Besonders auf die D\&D basierten eingehen, den rest nur als Beispiele für andere Systeme.

\section{Bestehende Ansätze?}
\label{sec:BekannteAnsaetze}
Paper und Programme die unserem Thema ähneln, jeweils mit Abgrenzung warum sie die Frage nicht zufriedenstellend (oder wenigstens schlechter als wir) beantworten können
\begin{itemize}
	\item RPG Maker
	\item NWN
	\item GM-AI
	\item Tools die ähnliches leisten
\end{itemize}