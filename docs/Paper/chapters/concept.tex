\chapter{Konzept}
\label{concept}
In den Abschnitten dieses Kapitels werden verschiedenen Elementare Konzepte vorgestellt, welche im Prototypen zum Einsatz kommen. Nach einer Übersicht über das Grobkonzept werden Kernelemente und Interaktive System erläutert um abschließend in \ref{sec:Spielfluss} das Zusammenspiel der einzelnen Systeme an einem kurzen fingierten Abenteuer aufzuzeigen.

\section{Spielkonzept}
\label{sec:Grobkonzept}
%\emph{Beschreibung des generellen Prototypen. Auf keinen Fall technisch.}
Der zur Beantwortung der Frage umgesetzte Prototyp benutzt eine abstrakte, zwei-dimensionale Darstellung um eine rudimentäre visuelle Repräsentation des Abenteuers bereit zu stellen. Diese abstrakte Darstellung wurde bewusst gewählt um die Spieler dazu anzuregen weiterhin eine eigenen Vorstellung der bespielten Welt zu pflegen. Mehrere Spieler (Clients) verbinden sich mit einer Spielleiter-Instanz (Server) über das Internet oder ein lokales Netzwerk. Der Prototyp implementiert verschiedene Komponenten welche das durchführen einer PnP Spielrunde mit, dem analogen Original ähnlicher, Stimmung und Spielfluss erlaubt.



\subsection{Kernelemente}

%\begin{itemize}
%	\item Abstrakte Tile based world
%	\item Spielleiter kann Welt/jedes Objekt editieren
%	\item Spieler haben Echtzeit Rollenspiel + rundenbasierten Kampfmodus
%	\item Annahme: TeamSpeak oder ähnliche effective Kommunikation
%	\item Clienten (Spieler),  Server (GM)
%	\item Objectsystem, Rechtesystem
%\end{itemize}
Hauptkomponente des Prototypen ist ein frei definierbares Objektsystem. Zu jedem Zeitpunkt steht es dem Spielleiter frei Objekte und Werte derartig zu verändern, dass ein völlig anderes Vorgehen ermöglicht wird. Denn intern behandelt der Prototyp alle Elemente als gleichwertige Objekte. Diesen Objekten können verschiedene Eigenschaften zugeordnet werden, welche das Verhalten der Objekte definieren. Dieses System erlaubt es dem Spielleiter jedes Objekt in der Spielwelt zu editieren und bei Bedarf mit weiteren Daten anzureichern. Schlägt ein Spieler vor einen Gegenstand aus der Umgebung als Waffe zu nutzen, so kann der Spielleiter dem entsprechenden Objekt einen Schadenswert hinzufügen und diesen später in der Berechnung des Kampfwertes nutzen.\newline
Durch Module und Schablonen werden dem Spielleiter vorgefertigte Kombinationen von Eigenschaften und verwendbaren Objekten geliefert, welche er bei Bedarf direkt einsetzen kann. Mit dieser Option kann er das Verhalten der Spielwelt schnell ändern, was die Einwirkung bei laufendem Spiel ermöglicht.\newline
Ein internes Rechtesystem erlaubt es Spielern nur einen gewissen Teils dieser Eigenschaften zu sehen oder diese ändern zu können. Auf diese Weise kann er zwar z.B. seine Trefferpunkte selbst verwalten, aber keine beliebigen Gegenstände verändern.
%NEE KANN ER LEIDER NICHT (gibt keine gui dafür). Über ein internes Rechtesystem kann der Spielleiter zudem definieren in welcher Weise die Objektdaten für Spieler sichtbar sind. So kann zum Beispiel einem Waldläufer eine Eigenschaft, die Daten über eine Fährte enthält, angezeigt werden, anderen Spielern bleibt diese Information jedoch verwehrt. Dies erlaubt es dem Spielleiter gezielt Daten für einzelne Spieler frei zu geben, oder aber auch Objekte wie NPCs oder Items mit Notizen anreichern, die nur für ihn selbst sichtbar sind.\newline

Der Prototyp liefert in einer Anwendung eine Server und eine Client Sicht aus. Der Spielleiter kann die Serveransicht nutzen um vor der Spielrunde das Abenteuer zu erstellen und leitet aus dieser Sicht heraus auch den Spielverlauf. Sobald der Server aktiviert wurde können sich die Spieler mit ihm verbinden und starten damit die Client-Sicht. Um eine ähnlich flüssige Kommunikation wie bei analogen PnP Runden zu ermöglichen wird davon ausgegangen, dass ein Echtzeitkommunikationsmedium wie z.B. Teamspeak\footnote{http://www.teamspeak.com/} oder Skype\footnote{http://www.skype.com/} genutzt wird. Zusätzlich bietet der Client noch einen einfachen Chat, über den Statusmeldungen, Spielernachrichten und Emotes ausgegeben werden können. \newline
Die Interaktion der Spieler mit der Spielwelt geschieht in Echtzeit, nur während eines Kampfes wechselt der Prototyp in das typische rundenbasierte Kampfsystem, welches auf den Dungeons\&Dragons Regelsystem basiert. Dem Spielleiter stehen gleichzeitig alle Optionen offen die Spielwelt im laufenden Spiel zu editieren.\newline
Durch die Kombination des offenen Objektsystems und der abstrakten Darstellung, welche schnell durch eigenen Grafiken erweitert oder verändert werden kann, bietet der Prototyp damit ein Hilfsmittel um eigene, persönliche Abenteuer zu erstellen und in einer PnP Runde zu bespielen.




\section{Interaktive Systeme}
\label{sec:InteraktiveSysteme}
Ein System veränderbarer Objekte erschafft zwar einen Baukasten, aber ein Spielfluss wird dadurch noch nicht realisiert. Das folgende Nachrichtensystem ist so allgemein gehalten, dass sich viele Aktionen direkt damit realisieren lassen. Das zusätzliche Kampfsystem setzt den häufig zentralen Teil des Kämpfens auf Basis des D\&D-Regelwerkes um.

\subsection{Kampfsystem}
\label{sec:Kampfsystem}
\todo{Kampfsystem stuff}

\subsection{Nachrichtensystem}
\label{sec:Nachrichtensystem}
Während der Prototyp nicht auf ein Echtzeitkommunikationsmedium wie z.B. Teamspeak verzichten kann bietet er mit seinem Nachrichtensystem eine umfassende Mechanik an um den Informationsfluss im Spiel zu steuern. 
Es lassen sich Nachrichten und auch einfache Emotes über den Chat versenden. Dies soll im Spiel eine gewissen Form des Rollenspiels ermöglichen um zum Beispiel Gefühlsregungen des Charakters zum Ausdruck zu bringen, während ein anderer Teilnehmer gerade spricht. Verschiedene Ereignisse wie Würfelergebnisse oder das Auslösen eines Schalters werden ebenso im Chat ausgegeben wie Fehler in eingegebenen Formeln. Der chat dient damit neben der Kommunikationsfunktion auch als Log für das Abenteuer.\newline
Zusätzlich zum Chat können verschiedene Aktionen von Spielern oder vom Spielleiter gefordert werden, zum Beispiel das Auswürfeln der Initiative zum Kampfbeginn. Solche angeforderten Aktionen werden mit ihrem jeweiligen Symbol am unteren Bildschirmrand angezeigt und können minimiert, bzw. aufgeklappt werden.\newline
Eine wichtige Aktion dieser Art ist die \emph{'Freitextaktion'}, welche von einem Spieler auf ein beliebiges Objekt angewendet werden kann. Der Spielleiter erhält daraufhin eine solche Meldung mit dem Ausführenden und dem Ziel der Aktion, sowie dem vom Spieler eingegebenen Freitext. Der Spielleiter hat dann direkt in der Nachricht die Möglichkeit zu den betroffenen Objekte zu springen um die Situation zu analysieren und entsprechend zu entscheiden.


\section{Spielfluss}
\label{sec:Spielfluss}
%Wird vermutlich eines der größten Kapitel im Konzept werden. Anhand der vorherigen Abschnitte erklären wie eine 'Spielsession' aus Sicht der Spieler und des Spielleiters ablaufen würde. Daran werden die Entscheidungen aus den vorherigen Abschnitten gerechtfertigt.\newline
%\todo{Überlegen ob wir das in einem Kapitel machen und Spieler / Spielleiter optisch abgrenzen können (Vorteil es ist zeitlich synchron, aber evtl. anstrengender zu lesen) oder ob wir in Spieler / Spielleiter Unterkapitel trennen (Hier wäre ganz kalr worum es grade geht, aber Ereignisse sind nicht mehr synchron. Alles in einem Fluss, verschiedene Schriftarten/Farben möglich; Auf jeden fall sehr deutlich. \textbf{Möglickeit: Logische Kapitel und dann Spielleiter/Spieler getrennt.}}


In diesem Kapitel wird die Funktionsweise und das Zusammenspiel der verschiedenen vorgestellten Systeme anhand eines kurzen Abenteuers aus Sicht der Spieler und des Spielleiters erläutert.\newline
Der Spielleiter, im folgenden auch \emph{SL} genannt beginnt damit ein Abenteuer zu konzipieren und vor zu bereiten. In diesem Szenario sollen die Spieler in einer Taverne von dem Wirt beauftragt werden seltsamen Geräuschen aus dem Keller nachzugehen. Im Keller finden die Spieler einen Durchbruch in ein tieferes Gewölbe und müssen sich dort dem Kampf mit zwei Goblins stellen, um am Ende mit einer Schatztruhe belohnt zu werden.\newline
\subsection{Vorbereitung}
\label{sec:Vorbereitung}
Der SL beginnt seine Vorbereite mit dem Layout der Umgebung. Dafür greift er auf verschiedene generische Tiles zu und nutzt die frei-definierbare Farbe um diverse Materialien darzustellen. Mit diesen Tiles erstellt er zwei verschiedene Karten, die Taverne und das Gewölbe, welches aus einem Eingangsbereich und einem größeren Raum für den Kampf und die Truhe besteht. Für die Gegner könnte der Spielleiter eine der vorbereiteten Schablonen nutzen, er entscheidet sich jedoch dafür einen eigenen Gegnertyp zu definieren. Dafür erstellt er ein neues Objekt und weist ihm eine Grafik zu. Danach fügt er das \emph{'Angreifbar'} Modul hinzu, welches die Eigenschaften \emph{'Trefferpunkte'} und \emph{'Rüstungsklasse'} enthält. Die Rüstungsklasse setzt der SL auf den fixen Wert 12 und die Trefferpunkte auf 15.\newline
Der SL platziert nun zwei Instanzen der grade erstellen Vorlage und eine Instanz der Truhen-Schablone in dem Verlies. Die Truhe besitzt eine Inventar Eigenschaft, die der Spielleiter nun mit einigen Münzen und einem Rüstungsobjekt befüllt. Zuletzt versiegelt der SL den Durchgang vom ersten in den zweiten Raum mit einigen bunten Balken, welche als Bücherregal fungieren werden. Einem Regal weißt der Spielleiter das 'Schalter' Modul zu, welches Spielern später die Möglichkeit der Interaktion gibt.\newline
Die Taverne füllt der Spielleitern mit einigen NPCs und Objekten, außerdem platziert er das Modul \emph{Sprungmarke} auf einer Tür im hinteren Bereich der Taverne und verbindet das Sprungziel mit dem ersten Raum der Gewölbe Karte.\newline
Die Vorbereitung des Abenteuers ist hiermit abgeschlossen.

\subsection{Einführung in das Abenteuer}
\label{sec:EinführungInDasAbenteuer}


Zur Spielrunde finden sich neben dem Spielleiter noch zwei Spieler (S1 und S2) ein. Nachdem beide Spieler mit einem an \emph{Pathfinder}\footnote{http://paizo.com/pathfinder}, eine Variante des D\&D-Regelwerkes, angelehnten Charakterbogen die Werte ihres Charakters definiert haben verbinden sie sich mit dem SL-Server. Der Prototyp synchronisiert nun die Spielwelt, Objektvorlagen und Medien des Spiels. Nachdem beide Spieler die Verbindung hergestellt haben platziert der Spielleiter sie in der Taverne.\newline
Die Spieler erkunden die Umgebung und bald entfaltet sich das Rollenspiel. 

\subsection{Rollenspiel}
\label{sec:Rollenspiel}

S1 fordert einen vom SL als besonders düster aussehend beschriebenen NPC zu einer Runde Armdrücken auf.\newline
Der Spielleiter würfelt verdeckt einen Wert für den NPC und fordert S1 auf ebenfalls einen Wurf mit einem W20 zuzüglich es Stärke-Modifikators auszuführen. Diese Eigenschaft ist an das Spielerobjekt angehängt und wird aus den Werten der Attribut Eigenschaften errechnet. S1 merkt an das ihr Charakter eine hübsche, verschlagene Diebin ist und möchte gerne ebenfalls den Charisma-Modifikator mit einbringen. Der SL genehmigt dies und S1 würfelt endgültig mit \emph{'W20 + ST-Mod + }'\emph{CH-Mod}'\emph{'}\newline
S1 gewinnt diesen Wurf und als Preis erhält sie eine Runde Getränke aufs Haus.\newline
Nach einigen Getränken und Gesprächen mit den NPCs, welche der SL über Teamspeak durchführt entscheiden sich die Spieler mit dem Abenteuer fortzuschreiten. 

\subsection{Das erste Rätsel}
\label{sec:DasErsteRätsel}

In einem Gespräch mit dem Inhaber erfahren S1 und S2 das es sonderbare Begebenheit im Keller der Taverne gibt. Der Spielleiter bewegt den Questgeber und gibt den Spielern so den Weg zur vorher definierten Sprungmarke frei.\newline
Voller Tatendrang und beschwingt von den feinen Getränken stürzen sich beide Spieler in das Abenteuer.
Nach der Interaktion mit der Sprungmarke werden beide Spieler automatisch auf der Gewölbekarte platziert. Der Spielleiter beschreibt die Umgebung kurz und lässt die Spieler dann die Umgebung selber erkunden. Bald hat S2 herausgefunden das es die Möglichkeit gibt mit einem der Bücherregale zu interagieren. Der SL erhält die Benachrichtigung darüber über den Chat und entfernt das Bücherregal, während er eindrucksvoll beschreibt wie es sich unter Ächzen und Knarren im Boden versenkt.\newline


\subsection{Der Kampf}
\label{sec:DerKampf}
Die Spieler betreten den zweiten Raum und der Spielleiter fordert sie sofort auf einen Wurf ihre Schleichfähigkeit auszuführen. Das Würfelglück ist nun jedoch nicht bei den Spielern und daher startet sofort ein Kampf mit den Goblins.\newline
Dafür selektiert der Spielleiter die Spieler und die Golbins und startet einen Kampf über seine GUI. S1 und S2 werden dadurch automatisch aufgefordert einen Initiative-Wurf abzulegen, der SL selbst würfelt diesen Wert für die Goblins aus. Es folgen einige Kampfrunden in denen Spieler und Goblins abwechselnd Angriffs und Schadenswürfe sowie Bewegungen ausführen.\newline
Der Kampf siegreich für die Spieler und sie können die Truhe im Raum untersuchen und sich an der reichen Beute bedienen. Der Spielleiter leitet die Spieler nun zurück in die Taverne und erzählt einen kurzen Abschluss des Abenteuers.


