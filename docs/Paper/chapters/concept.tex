\chapter{Konzept}
\label{concept}


\section{Spielkonzept}
\label{sec:Grobkonzept}
%\emph{Beschreibung des generellen Prototypen. Auf keinen Fall technisch.}
Der zur Beantwortung der Frage umgesetzte Prototyp benutzt eine abstrakte, zwei-dimensionale Darstellung um eine rudimentäre visuelle Repräsentation des Abenteuers bereit zu stellen. Diese abstrakte Darstellung wurde bewusst gewählt um die Spieler dazu anzuregen weiterhin eine eigenen Vorstellung der bespielten Welt zu pflegen. Mehrere Spieler (Clients) verbinden sich mit einer Spielleiter-Instanz (Server) über das Internet \todo{drin lassen?} oder ein lokales Netzwerk. Der Prototyp implementiert verschiedene Komponenten welche das durchführen einer PnP Spielrunde mit, dem analogen Original ähnlicher, Stimmung und Spielfluss.



\subsection{Kernelemente}

\begin{itemize}
	\item Abstrakte Tile based world
	\item Spielleiter kann Welt/jedes Objekt editieren
	\item Spieler haben Echtzeit Rollenspiel + rundenbasierten Kampfmodus
	\item Annahme: TeamSpeak oder ähnliche effective Kommunikation
	\item Clienten (Spieler),  Server (GM)
	\item Objectsystem, Rechtesystem
\end{itemize}

Der Prototyp liefert in einer Anwendung eine Server und eine Client Sicht aus. Der Spielleiter kann die Serveransicht nutzen um vor der Spielrunde das Abenteuer zu erstellen und leitet aus dieser Sicht heraus auch den Spielverlauf. Sobald der Server aktiviert wurde können sich die Spieler mit ihm verbinden und starten damit die Client-Sicht. Um eine ähnlich flüssige Kommunikation wie bei analogen PnP Runden zu ermöglichen wir davon ausgegangen, dass ein Echtzeitkommunikationsmedium wie z.B. Teamspeak oder Skype genutzt wird. Zusätzlich bietet der Client noch einen einfachen Chat, über den Statusmeldungen, Spielernachrichten und Emotes ausgegeben werden können. 

\section{Interaktive Systeme}
\label{sec:InteraktiveSysteme}
Übersicht über die Systeme die wir allgemeingültig implementieren\newline
Am Beispiel des Kampfsystems (evaluieren von Formeln, Einfluss von Spieler etc.)\newline
Nachrichtensystem (Events an Spielleiter, Chat etc)\newline


\section{Spielfluss}
\label{sec:Spielfluss}
Wird vermutlich eines der größten Kapitel im Konzept werden. Anhand der vorherigen Abschnitte erklären wie eine 'Spielsession' aus Sicht der Spieler und des Spielleiters ablaufen würde. Daran werden die Entscheidungen aus den vorherigen Abschnitten gerechtfertigt.\newline
\todo{Überlegen ob wir das in einem Kapitel machen und Spieler / Spielleiter optisch abgrenzen können (Vorteil es ist zeitlich synchron, aber evtl. anstrengender zu lesen) oder ob wir in Spieler / Spielleiter Unterkapitel trennen (Hier wäre ganz kalr worum es grade geht, aber Ereignisse sind nicht mehr synchron. Alles in einem Fluss, verschiedene Schriftarten/Farben möglich; Auf jeden fall sehr deutlich. \textbf{Möglickeit: Logische Kapitel und dann Spielleiter/Spieler getrennt.}}


In diesem Kapitel wird anhand eines kurzen Abenteuers der Spielverlauf aus Sicht der Spieler und des Spielleiters erläutert.