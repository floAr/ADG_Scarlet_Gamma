\chapter{Konzept}
\label{concept}


\section{Spielkonzept}
\label{sec:Grobkonzept}
Beschreibung des generellen Prototypen. Auf keinen Fall technisch.

\subsection{Säulen}

\begin{itemize}
	\item Abstrakte Tile based world
	\item Spielleiter kann Welt/jedes Objekt editieren
	\item Spieler haben Echtzeit Rollenspiel + rundenbasierten Kampfmodus
	\item Annahme: TeamSpeak oder ähnliche effective Kommunikation
	\item Clienten (Spieler),  Server (GM)
	\item Objectsystem, Rechtesystem
\end{itemize}



\section{Interaktive Systeme}
\label{sec:InteraktiveSysteme}
Übersicht über die Systeme die wir allgemeingültig implementieren\newline
Am Beispiel des Kampfsystems (evaluieren von Formeln, Einfluss von Spieler etc.)\newline
Nachrichtensystem (Events an Spielleiter, Chat etc)\newline


\section{Spielfluss}
\label{sec:Spielfluss}
Wird vermutlich eines der größten Kapitel im Konzept werden. Anhand der vorherigen Abschnitte erklären wie eine 'Spielsession' aus Sicht der Spieler und des Spielleiters ablaufen würde. Daran werden die Entscheidungen aus den vorherigen Abschnitten gerechtfertigt.\newline
\todo{Überlegen ob wir das in einem Kapitel machen und Spieler / Spielleiter optisch abgrenzen können (Vorteil es ist zeitlich synchron, aber evtl. anstrengender zu lesen) oder ob wir in Spieler / Spielleiter Unterkapitel trennen (Hier wäre ganz kalr worum es grade geht, aber Ereignisse sind nicht mehr synchron. Alles in einem Fluss, verschiedene Schriftarten/Farben möglich}
