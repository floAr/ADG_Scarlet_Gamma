\chapter{Konzept}
\label{concept}


\section{Grobkonzept}
\label{sec:Grobkonzept}
Beschreibung des generellen Prototypen

\begin{itemize}
	\item Client
	\item Server
	\item Usecases (für Spieler und Spielleiter)s
\end{itemize}



\subsection{Objekt- \& Eigenschaften-System}
\label{sec:ObjektEigenschaftenSystem}

\begin{itemize}
	\item Alles sind Objekte
	\item Objekte beinhalten Eigenschaften
\end{itemize}


\subsection{Kommunikation}
\label{sec:Kommunikation}


\begin{itemize}
	\item synchronisierung über Netzwerk
	\item kurze beschreibung 
	\item \todo{Denke das ist kein eigenes Kapitel, ist zwar viel was wir machen, aber im Endeffekt nicht relevant für die Lösung der Frage}
\end{itemize}


\section{Objektsystem}
\label{sec:Objektsystem}
Genauere Beschreibung unseres Objektsystems\newline


\subsection{Eigenschaften}
\label{sec:Eigenschaften}

\begin{itemize}
	\item Erklärung von Eigenschaften
	\item Beispieleigenschaften mit Funktionalität 
\end{itemize}


\subsection{Rechtesystem}
\label{sec:Rechtesystem}
Beschreibung der Berechtigungen



\section{Interaktive Systeme}
\label{sec:InteraktiveSysteme}
Übersicht über die Systeme die wir allgemeingültig implementieren\newline
Am Beispiel des Kampfsystems (evaluieren von Formeln, Einfluss von Spieler etc.)\newline
Nachrichtensystem (Events an Spielleiter, Chat etc)\newline


\section{Spielfluss}
\label{sec:Spielfluss}
Wird vermutlich eines der größten Kapitel im Konzept werden. Anhand der vorherigen Abschnitte erklären wie eine 'Spielsession' aus Sicht der Spieler und des Spielleiters ablaufen würde. Daran werden die Entscheidungen aus den vorherigen Abschnitten gerechtfertigt.\newline
\todo{Überlegen ob wir das in einem Kapitel machen und Spieler / Spielleiter optisch abgrenzen können (Vorteil es ist zeitlich synchron, aber evtl. anstrengender zu lesen) oder ob wir in Spieler / Spielleiter Unterkapitel trennen (Hier wäre ganz kalr worum es grade geht, aber Ereignisse sind nicht mehr synchron.}
