\chapter{Zusammenfassung}
\label{conclusion}
In dieser Ausarbeitung wurde auf Basis des gängigen Dungeons-\&-Dragons-Regel\-werks ein Konzept erarbeitet, das die Prinzipien des Pen-\&-Papers-Rollenspiels auf digitale Rollenspiele überträgt. Zunächst wurden digitale Rollenspiele mit PnP-Rollenspielen verglichen und die Defizite aufgezeigt. Die Kernelemente von PnP-Rollenspielen wurden identifiziert und während der Erarbeitung des Konzeptes priorisiert.\\

\section{Fazit}
Der konzipierte Prototyp wurde in \ref{sec:Auswertung} anhand der vordefinierten Erfolgskriterien im Probandentest evaluiert. Die Auswertung der Tests ergab, dass alle Erfolgskriterien erfüllt werden konnten. Spieler, die bereits ERfahrungen in PnP- und digitalen Rollenspielen hatten, empfanden die Handlungfreiheit des Prototypen als nicht eingeschränkt. Trotz dieses komplexen und offenen System erlaubt es der Prototyp weiterhin schnell und frei von störenden Verzögerungen zu arbeiten. Spielleiter sind in der Lage nach kurzer Einarbeitung und durch Zuhilfenahme des Handbuches komplexe Probleme eines PnP-Abenteuers in dem Prototypen zu modellieren und zu lösen. Diese Qualität des Prototypen zeigt sich auch in den Fragen 9 und 10 von \emph{Fragebogen A}: Ein Großteil der Tester würde den Prototypen gerne als digitales Hilfsmitteln in privaten PnP-Runden einsetzen. Insbesondere das Kampfsystem, welches durch die abstrakte, leicht zu erstellende Welt und das intuitive Setzen und Bewegen der Figuren ein schnelles und einfaches Hilfsmittel darstellt.
\todo{noch kein ersatz weil zu viele regeln fehlen}


%In dieser Ausarbeitung wurde auf Basis des gängigen Dungeons-\&-Dragons-Regel\-werks ein Konzept erarbeitet, das die Prinzipien des Pen-\&-Papers-Rollenspiels auf digitale Rollenspiele überträgt. Zunächst wurden digitale Rollenspiele mit PnP-Rollenspielen verglichen und die Defizite aufgezeigt. Die Kernelemente von PnP-Rollenspielen wurden identifiziert und während der Erarbeitung des Konzeptes priorisiert.\\
%Anhand eines Prototyps wurde das Konzept umgesetzt und im Probandentest evaluiert. Die Auswertung der Tests ergab zunächst, dass unser Prototyp intuitiv und effizient nutzbar und somit geeignet ist, um die Fragestellung zu beantworten. Weiterhin konnte in den Tests eine erfolgreiche Übertragung der PnP-Prinzipien auf digitale Rollenspiele bestätigt werden. Die Fragestellung, die den Kern dieser Ausarbeitung darstellt, wurde somit bestätigt.

\section{Zukünftige Arbeiten}
\label{sec:futurework}

Der Prototyp konnte zeigen, dass es möglich ist, ein PC-RPG zu erschaffen welches die Handlungsfreiheit der Spieler nicht einschränkt. Ein völlig freies System verliert jedoch oftmals an Komfort, den andere Systeme durch das Treffen gewisser Annahmen gewinnen können. In unserem Konzept wurde dieses Problem dadurch umgangen, dass der Spielleiter stets uneingeschränkt handeln kann und Komfort-Mechanismen wie z.B. das Kampfsystem verwenden \emph{kann}, aber nicht \emph{muss}. Dieser Komfort ist jedoch ein entscheidender Faktor, den digitale Rollenspiele den PnP-Rollenspielen voraus haben. In unserem Prototyp haben wir nur einen Bruchteil der Regeln von Dungeons \& Dragons implementiert. Die Implementierung des kompletten Regelwerks sprengt bei Weitem den Rahmen dieser Ausarbeitung, wäre jedoch nun, da unsere Fragestellung positiv beantwortet werden konnte, ein angemessener nächster Schritt.

\todo{Wir haben konkrete Tickets (Zauber, Objekte größer als 1x1 Tile, Waffenslots und besseres Inventar, mehr Chat-Funktionen) die jedoch alle, wenn man den Text liest, nicht wirklich relevant werden. Trotzdem rein?}