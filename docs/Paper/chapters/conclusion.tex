\chapter{Zusammenfassung}
\label{conclusion}
In dieser Ausarbeitung wurde auf Basis des gängigen Dungeons-\&-Dragons-Regel\-werks ein Konzept erarbeitet, das die Prinzipien des Pen-\&-Papers-Rollenspiels auf digitale Rollenspiele überträgt. Zunächst wurden digitale Rollenspiele mit PnP-Rollenspielen verglichen und die Defizite aufgezeigt. Die Kernelemente von PnP-Rollenspielen wurden identifiziert und während der Erarbeitung des Konzeptes priorisiert.\\

\section{Fazit}
Der konzipierte Prototyp wurde in \ref{sec:Auswertung} anhand der vordefinierten Erfolgskriterien im Probandentest evaluiert. Die Auswertung der Tests ergab, dass alle Erfolgskriterien erfüllt werden konnten. Spieler, die bereits Erfahrungen in PnP- und digitalen Rollenspielen hatten, empfanden die Handlungsfreiheit des Prototypen als nicht eingeschränkt. Trotz dieses komplexen und offenen System erlaubt es der Prototyp weiterhin schnell und frei von störenden Verzögerungen zu arbeiten. Spielleiter sind in der Lage nach kurzer Einarbeitung und durch Zuhilfenahme des Handbuches komplexe Probleme eines PnP-Abenteuers in dem Prototypen zu modellieren und zu lösen. Diese Qualität des Prototypen zeigt sich auch in den Fragen 9 und 10 von \emph{Fragebogen A}: Ein Großteil der Tester würde den Prototypen gerne als digitales Hilfsmitteln in privaten PnP-Runden einsetzen. Insbesondere das Kampfsystem, welches durch die abstrakte, leicht zu erstellende Welt und das intuitive Setzen und Bewegen der Figuren ein schnelles und einfaches Hilfsmittel darstellt.\newline
Um das Programm alleinstehend für PnP-Runden einzusetzen fehlt jedoch noch zu viel des Regelsystems. Testspieler äußerten Interesse eine Runde komplett mit dem Prototypen durch zu spielen, forderten jedoch dafür, dass alle Dungeon-\&.Dragons-Regeln vollständig implementiert sein müssen.


%In dieser Ausarbeitung wurde auf Basis des gängigen Dungeons-\&-Dragons-Regel\-werks ein Konzept erarbeitet, das die Prinzipien des Pen-\&-Papers-Rollenspiels auf digitale Rollenspiele überträgt. Zunächst wurden digitale Rollenspiele mit PnP-Rollenspielen verglichen und die Defizite aufgezeigt. Die Kernelemente von PnP-Rollenspielen wurden identifiziert und während der Erarbeitung des Konzeptes priorisiert.\\
%Anhand eines Prototyps wurde das Konzept umgesetzt und im Probandentest evaluiert. Die Auswertung der Tests ergab zunächst, dass unser Prototyp intuitiv und effizient nutzbar und somit geeignet ist, um die Fragestellung zu beantworten. Weiterhin konnte in den Tests eine erfolgreiche Übertragung der PnP-Prinzipien auf digitale Rollenspiele bestätigt werden. Die Fragestellung, die den Kern dieser Ausarbeitung darstellt, wurde somit bestätigt.

\section{Zukünftige Arbeiten}
\label{sec:futurework}

Der Prototyp konnte zeigen, dass es möglich ist, ein PC-RPG zu erschaffen welches die Handlungsfreiheit der Spieler nicht einschränkt. Ein völlig freies System verliert jedoch oftmals an Komfort, den andere Systeme durch das Treffen gewisser Annahmen gewinnen können. In unserem Konzept wurde dieses Problem dadurch umgangen, dass der Spielleiter stets uneingeschränkt handeln kann und Komfort-Mechanismen wie z.B. das Kampfsystem verwenden \emph{kann}, aber nicht \emph{muss}. Dieser Komfort ist jedoch ein entscheidender Faktor, den digitale Rollenspiele den PnP-Rollenspielen voraus haben. In unserem Prototyp haben wir nur einen Bruchteil der Regeln von Dungeons \& Dragons implementiert. Die Implementierung des kompletten Regelwerks sprengt bei Weitem den Rahmen dieser Ausarbeitung, wäre jedoch nun, da unsere Fragestellung positiv beantwortet werden konnte, ein angemessener nächster Schritt.\newline
Effektiv setzt der Prototyp nur Regeln für eine grundlegende Bewegung zu Fuß und das Kampfsystem um und unterstützt würfelbasierte Regeln. Weitere Regeln, die man in das System integrieren könnte, sind zusätzliche Fortbewegungsarten, Zauber, automatische Proben, sowie die Charaktererstellung. Wie einige Regeln in unserem Konzept umgesetzt werden könnten wird im folgenden Beispielhaft erläutert.\newline
Derzeit müssen die Spieler Zauber manuell verwalten und der Schaden im Kampf manuell verrechnet werden. Hier könnten Zauberslots als Eigenschaften hinzugefügt und während des Kampfes als zusätzliche Option angeboten werden. Die Zauber selbst könnten rekursiv durch Objekte beschrieben werden, welche Eigenschaften wie '\textit{Schaden}' enthalten und beim Spieler im entsprechenden Zauberslot gelistet werden.\newline
Damit ein Spieler schwimmen oder fliegen kann könnte eine weitere Eigenschaft '\textit{Fortbewegung}' eingeführt werden, die diesen Status anzeigt. Der Weg-such-Algorithmus müsste diese Eigenschaft beachten und spezialisierte '\textit{Hindernis}'-Eigenschaften unterschiedlich interpretieren. Wenn ein Spieler nun die Aktion schwimmen verwendet könnte eine automatische Probe ausgeführt werden.\newline
Eine mögliche Umsetzung für automatische Proben, welche einen hohen Umfang an Dynamik erlauben würde ist die Einführung neuer Spezialeigenschaften. Es wäre sinnvoll, wenn an einem Objekt oder einer Eigenschaft Proben angehängt werden könnten. Ausgelöst werden diese bei Interaktion oder betreten des gleichen Feldes, wodurch zusätzlich Optionen wie das bauen von Fallen erlaubt werden würden.\newline
Ebenso wie Kernmechaniken können weitere Komfort-Features umgesetzt werden. Beispielhaft seien genannt:
ein Lineal zur Distanzmessung,
Autovervollständigung im Chat,
Referenzierung von Objekten und Orten im Chat,
ein strukturierter Inventarbildschirm und
die Erzeugung zufälliger Werte beim Erstellen von Objekten aus Schablonen, z.B. bei Trefferpunkten.\newline
Alle oben genannten Vorschläge setzen den Weg fort PnP Spiele direkt auf den PC zu übertragen. Es wäre ebenfalls interessant das Konzept eines menschlichen Spielleiters in andere Spiele zu übertragen. Das Hauptproblem welches wir dabei sehen ist, dass Spieler bei unbegrenzten Möglichkeiten unbegrenzt schummeln können. Diese Problem trat bei uns nicht auf, da PnP Gruppen üblicherweise ohnehin das nötige Vertrauen in all Beteiligten setzen.