\chapter{Evaluierung}
\label{evaluation}
%Die Beurteilung ist einer der wichtigsten Abschnitte der Arbeit
%- Sie enthält die Quintessenz des gesamten Projektes
%Viele lesen nur die Einführung und die Beurteilung an
%- Hier muss also alles Wichtige drin stehen!
%Hier beweisen Sie dass Sie …
%- die Aufgabe und deren Bedeutung verstanden haben
%- die Ergebnisse richtig zu interpretieren vermögen
%- wissen, worauf es bei diese Arbeit ankam


\section{Erfolgskriterien}
\label{sec:Erfolgskriterien}
Um die Fragestellung zu beantworten wird ein Probandentest mit Teilnehmern, die bereits Erfahrung mit Pen-\&-Paper-Rollenspiel gemacht haben, durchgeführt. Folgende Kriterien dienen am Ende des Projektes zur
Auswertung der Frage:

\begin{enumerate}
	\item Mindestens 66\% der Testspieler fühlen sich in ihrem Handlungsfreiraum nicht eingeschränkt. Im Probandentest soll ermittelt werden, ob Spieler sich subjektiv in ihrem Handlungsfreiraum eingeschränkt fühlen, also ob sie Aktionen ausführen möchten, auf die der Spielleiter nicht reagieren kann.
	\item Höchstens 33\% der Testspieler (inkl. Spielleiter) empfanden die Wartezeiten bei der Umsetzung ihrer Ideen durch den Spielleiter als zu lang. Im Probandentest soll ermittelt werden, ob der Spielleiter auf jede Situation in angemessener Zeit reagieren kann. Dies bedeutet, dass nur wenige Spieler in einer anschließenden Befragung angeben, durch lange Wartezeiten gestört worden zu sein.
	\item Der Spielleiter begreift die Standardfunktionen intuitiv und arbeitet eine Menge von Aktionen in vertretbarer Zeit ab. Der Spielleiter soll in der Lage sein, alle Funktionen, die das Spiel bietet (Objekte erstellen, platzieren, modifizieren, auf Spieleraktionen reagieren etc.), schnell zu begreifen und zu verwenden. Hierfür wird ein Satz von Aufgaben festgelegt, der diese Funktionen beinhaltet. Die Aufgaben werden innerhalb festgelegter Zeiträume abgeschlossen. Die jeweilige Dauer wird mit Konkretisierung der Aufgabenstellung bestimmt.
\end{enumerate}

Dieser Probandentest wird über zwei, im \hyperref[sec:frageboegen]{Anhang referenzierte}, Fragebögen durchgeführt. Fragebogen A wird dabei von allen Teilnehmern ausgefüllt um ein Meinungsbild über verschiedenen Aspekte der Erfolgskriterien zu erhalten. Fragebogen B wird speziell von Spielleitern bearbeitet. B enthält einige Aufgaben die häufiger in der Vorbereitung, bzw. Spielphase einer PnP Runde auftreten. Durch das Messen der benötigten Zeit soll ein Messwert generiert werden, der auf die Arbeitsgeschwindigkeit von Spielleitern schließen lässt.
  



\section{Testaufbau}
\label{sec:Testaufbau}
\begin{itemize}
	\item Anzahl der Testspiele
	\item Anzahl Spieler
	\item Anzahl unterschiedlicher Spieler / Spielleiter
	\item Genaue Aufgabenstellung der Tests (Appendix: genau der Fragebogen, den wir rausgeben)
\end{itemize}
Die Datenerhebung dieser Evaluierung wurde über einen Fragebogen durchgeführt, welcher während eines Probandentest von den Teilnehmern ausgefüllt wurde. Der Fragebogen enthält zehn Fragen, die dazu dienen die Rollenspielerfahrung der Testperson, sowohl digital als auch PnP, zu bestimmen (Frage 1 -3) und ein Qualitätslevel des Prototypen zu evaluieren (Frage 4-10). Die Qualitätsfragen zahlen dabei auf die verschiedenen Erfolgskriterien ein. Insgesamt wurde die Auswertungen in \todo{Anzahl Testrunden} durchgeführt, wobei \todo{Anzahl Spieler} und \todo{Anzahl Spielleiter} die Fragen ausgefüllt haben. Alle befragten Spieler hatten bereits umfassende Erfahrungen mit PnP Rollenspielen als Spieler und über zwei drittel haben ebenfalls schon als Spielleiter agiert. Auch bei digitalen Rollenspiele gaben alle Probanden umfassende Kenntnisse an.



\section{Auswertung}
\label{sec:Auswertung}

In diesem Abschnitt werden die erhobenen Daten in Bezug gesetzt und anhand dieser die Erfolgskriterien evaluiert.


\subsection{Handlungsfreiheit}
\label{sec:Handlungsfreiheit}
Der wohl elementarste Faktor des Prototypen war die Handlungsfreiheit von Spielern und Spielleiter. Dieser Aspekt wird durch die Fragen 4 und 5 auf \emph{Fragebogen A} evaluiert. Zum einen wurde beleuchtet, ob Spieler und Spielleiter sich in der Lage fühlten ihre eigenen Ideen im Spielfluss umzusetzen und damit ihre Kreativen Lösungsstrategien einzubringen, zum anderen wurde ausgeschlossen, dass das Spielsystem den Handlungsfreiraum der Tester einschränkt. 		
