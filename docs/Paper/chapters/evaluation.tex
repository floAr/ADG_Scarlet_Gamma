\chapter{Evaluierung}
\label{evaluation}
%Die Beurteilung ist einer der wichtigsten Abschnitte der Arbeit
%- Sie enthält die Quintessenz des gesamten Projektes
%Viele lesen nur die Einführung und die Beurteilung an
%- Hier muss also alles Wichtige drin stehen!
%Hier beweisen Sie dass Sie …
%- die Aufgabe und deren Bedeutung verstanden haben
%- die Ergebnisse richtig zu interpretieren vermögen
%- wissen, worauf es bei diese Arbeit ankam


\section{Erfolgskriterien}
\label{sec:Erfolgskriterien}
Um die Fragestellung zu beantworten wird ein Probandentest mit Probanden, die bereits ein Pen-\&-
Paper-Rollenspiel gespielt haben, durchgeführt. Folgende Kriterien dienen am Ende des Projektes zur
Beantwortung der Frage:

\begin{itemize}
	\item Mindestens 66\% der Testspieler fühlen sich in ihrem Handlungsfreiraum nicht eingeschränkt
Im Probandentest soll ermittelt werden, ob Spieler sich subjektiv in ihrem Handlungsfreiraum
eingeschränkt fühlen, also ob sie Aktionen ausführen möchten, auf die der Spielleiter nicht
reagieren kann.
	\item Höchstens 33\% der Testspieler (inkl. Spielleiter) empfanden die Wartezeiten bei der
Umsetzung ihrer Ideen durch den Spielleiter als zu lang
Im Probandentest soll ermittelt werden, ob der Spielleiter auf jede Situation in
angemessener Zeit reagieren kann. Dies bedeutet, dass nur wenige Spieler in einer
anschließenden Befragung angeben, durch lange Wartezeiten gestört worden zu sein.
	\item Der Spielleiter soll die Standardfunktionen intuitiv begreifen und eine Menge von Aktionen in
vertretbarer Zeit abarbeiten können
Der Spielleiter soll in der Lage sein, alle Funktionen, die das Spiel bietet (Objekte erstellen,
platzieren, modifizieren, auf Spieleraktionen reagieren etc.), schnell zu begreifen und zu
verwenden. Hierfür wird ein Satz von Aufgaben festgelegt, die diese Funktionen beinhaltet.
Die Aufgaben sollen innerhalb festgelegter Zeiträume abgeschlossen werden. Die jeweilige
Dauer wird mit Konkretisierung der Aufgabenstellung bestimmt.
\end{itemize}
  



\section{Testaufbau}
\label{sec:Testaufbau}
Genaue Beschreibung des Tests

\begin{itemize}
	\item Anzahl der Testspiele
	\item Anzahl Spieler
	\item Anzahl unterschiedlicher Spieler / Spielleiter
	\item Genaue Aufgabenstellung der Tests
\end{itemize}


\section{Auswertung}
\label{sec:Auswertung}
Ergenisse der Tests, orinetiert an den Erfolgskriterien (vermutlich ein Unterkapitel für jedes Erfolgskrit.)