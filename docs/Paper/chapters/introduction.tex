\ifgerman{\chapter{Einführung}}{\chapter{Introduction}}
%- Hintergrund
%- Motivation
%- Ziele
%- Aufgaben
%- Allgemeine Beschreibung des Projektes
%- Worum geht es in dieser Arbeit?
%- Wer hat die Arbeit veranlasst und wozu?
%- Wer soll von den Ergebnissen profitieren?
%- Welches Problem soll gelöst werden? Warum?
%- Unter welchen Umständen braucht man eine Verbesserung?
%- Was ist der Stand der Technik?
%- Welche noch offenen Probleme gibt es?
%- Worin unterscheidet sich mein Ansatz von den bisherigen?
%- Welche Ziele hat die Arbeit?
%- Wie will ich diese Ziele erreichen?
%- Was habe ich im Einzelnen vor?


\section{Motivation}
\label{sec:Motivation}

\section{Wissenschaftliche Frage}
\label{sec:WissenschaftlicheFrage}

Pen-\&-Paper-Rollenspiele basieren auf komplexen Regelwerken, finden jedoch vollständig in der Fantasie der Spieler statt. Hierbei übernimmt ein Spieler die Rolle des Spielleiters, der die Geschichte der Welt erzählt und die Regeln auslegt und erweitert um den Spielern ein möglichst freies und umfassendes Spielerlebnis zu bieten. Obwohl diverse PC-Rollenspiele auf diesen Regelsystemen beruhen, ist die im Pen-\&-Paper übliche Freiheit nicht gegeben. Die „Spielleiter“ sind in diesem Falle die Entwickler, die jedoch vom Spielerlebnis getrennt sind und nur eine begrenzte Anzahl verschiedener Möglichkeiten bedenken können. Eine Reaktion in Echtzeit auf die Ideen der Spieler ist in diesen Spielen nicht mehr möglich.\newline
Gerade diese Freiheit ist es jedoch, die für viele Spieler den Reiz von Pen-\&-Paper-Rollenspielen ausmacht. Ein gleichzeitig agierender Spielleiter könnte auf unvorhergesehen Aktionen der Spieler mit menschlicher Intelligenz reagieren und damit ein immersiveres Spielerlebnis schaffen, welches unbegrenzte Möglichkeiten suggeriert. 
Im Rahmen dieser Arbeit soll eine Antwort auf folgende wissenschaftliche Fragestellung gefunden werden:\newline
\emph{Kann ein Multiplayer-RPG-Spiel entwickelt werden, bei dem ein Spielleiter die Regeln des Spiels erweitert, so dass der Handlungsfreiraum der Spieler nicht behindert wird?}

\section{Zielstellung}
\label{sec:Zielstellung}

Zur Erörterung dieser Frage wird ein Prototyp umgesetzt, an dem die verschiedenen Aspekte der in \ref{sec:Zielstellung} formulierten Fragestellung evaluiert werden sollen. Hierbei wird sowohl der Bereich der wahrgenommen Handlungsfreiheit der Spieler, als auch die intuitive Bedienung durch den Spielleiter beleuchtet. In \ref{sec:Erfolgskriterien} werden diese Kriterien genauer quantifiziert und anhand von Nutzerstudien evaluiert.

\section{Gliederung}
\label{sec:Gliederung}

