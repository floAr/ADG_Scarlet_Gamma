\chapter{Einführung}
%- Hintergrund
%- Motivation
%- Ziele
%- Aufgaben
%- Allgemeine Beschreibung des Projektes
%- Worum geht es in dieser Arbeit?
%- Wer hat die Arbeit veranlasst und wozu?
%- Wer soll von den Ergebnissen profitieren?
%- Welches Problem soll gelöst werden? Warum?
%- Unter welchen Umständen braucht man eine Verbesserung?
%- Was ist der Stand der Technik?
%- Welche noch offenen Probleme gibt es?
%- Worin unterscheidet sich mein Ansatz von den bisherigen?
%- Welche Ziele hat die Arbeit?
%- Wie will ich diese Ziele erreichen?
%- Was habe ich im Einzelnen vor?


\section{Motivation}
\label{sec:Motivation}

Pen-\&-Paper-Rollenspiele bilden eine spielerische Adaption von geschriebener Fiktion und zählen zu den erfolgreichsten Implementierungen interaktiver Erzählung \cite{Tychsen2006}. Regelsysteme wie das 1974 veröffentliche \emph{Dungeon and Dragons} unterscheiden sich insofern von ihren literarischen Vorlagen, dass sie weniger eine festgeschriebene Handlung als ein Regelwerk zur Interaktion der Spieler mit der Spielwelt vorgeben \cite{Apperley2006}. Die Handlung wird von einem menschlichen Spielleiter im Dialog mit den Spieler selbst erschaffen.

Computer-basierte Rollenspiele verwenden zwar meist ähnliche Regelsysteme, jedoch werden die sozialen und kreativen Aspekte stark reduziert: die ersten Rollenspiele waren reine Einzelspieler-Kampagnen, in denen der Computer die Rolle des Spielleiters übernimmt \cite{Apperley2006}. Diese Entwicklung führte zu einer Verschiebung der Prioritäten im Spiel. Von der interaktiven Ausgestaltung der Geschichte wanderte der Fokus auf eine Fortentwicklung des Charakters \todo{quelle: The nature of computer games: Play as semiosis}.

Der Aspekt der freien Gestaltung einer Handlung während des Spielverlaufes ist in aktuellen PC-Rollenspielen nahezu nicht mehr vorhanden. Diese Arbeit beschäftigt sich mit der Frage, inwieweit diese Eigenschaft aus dem Medium des Pen-\&-Paper-Rollenspiels auf das digitale PC-Rollenspiel übertragbar ist. Hierzu soll einem menschlichen Spielleiter die Möglichkeit geben werden, so auf das Spielgeschehen einzugreifen und die Regeln zu ändern, dass die Spieler die Handlung und Welt wie in der Pen-\&-Paper-Variante beeinflussen können.

\section{Wissenschaftliche Frage}
\label{sec:WissenschaftlicheFrage}

Pen-\&-Paper-Rollenspiele basieren auf komplexen Regelwerken, finden jedoch nahezu vollständig in der Fantasie der Spieler statt. Hierbei übernimmt ein Spieler die Rolle des Spielleiters, der die Geschichte der Welt erzählt und die Regeln auslegt und erweitert um den Spielern ein möglichst freies und umfassendes Spielerlebnis zu bieten. Obwohl diverse PC-Rollenspiele auf diesen Regelsystemen beruhen, ist die im Pen-\&-Paper übliche Freiheit nicht gegeben. Die „Spielleiter“ sind in diesem Falle die Entwickler, die jedoch vom Spielerlebnis getrennt sind und nur eine begrenzte Anzahl verschiedener Möglichkeiten bedenken können. Eine Reaktion in Echtzeit auf die Ideen der Spieler ist in diesen Spielen nicht mehr möglich. Gerade diese Freiheit ist es jedoch, die für viele Spieler den Reiz von Pen-\&-Paper-Rollenspielen ausmacht.

Ein gleichzeitig agierender Spielleiter könnte auf unvorhergesehen Aktionen der Spieler mit menschlicher Intelligenz reagieren und damit ein immersives Spielerlebnis schaffen, welches unbegrenzte Möglichkeiten suggeriert.
Im Rahmen dieser Arbeit soll eine Antwort auf folgende wissenschaftliche Fragestellung gefunden werden:
\vspace*{0.5em}\begin{center}\parbox{0.9\linewidth}{
  \emph{Kann ein Multiplayer-RPG-Spiel entwickelt werden, bei dem ein Spielleiter die Regeln während des Spielens erweitert, so dass der Handlungsfreiraum der Spieler nicht behindert wird?}
} \end{center}\vspace*{0.5em}

\section{Zielstellung}
\label{sec:Zielstellung}

Zur Erörterung der Fragestellung wird ein Prototyp umgesetzt, an dem die verschiedenen Aspekte der in \ref{sec:WissenschaftlicheFrage} formulierten Fragestellung evaluiert werden sollen. Anhand dieses Prototypen soll überprüft werden, ob es möglich ist ein simples Multiplayer-RPG zu kreieren, welches die in der Frage geforderten Aspekte erfüllt. Darunter fällt vor allen die Möglichkeit der Spieler sich kreativ in der Spielwelt ausleben zu können. Damit verbunden ist ein System, welches dem Spielleiter eine Anpassung bzw. Erweiterung des internen Regelwerks ermöglicht, um eben diese freien Aktionen der Spieler zu ermöglichen. Hierbei wird sowohl der Bereich der wahrgenommen Handlungsfreiheit der Spieler, als auch die intuitive Bedienung durch den Spielleiter beleuchtet.

\subsection{Erfolgskriterien}
\label{sec:Erfolgskriterien_ziel}

Die Erfüllung der Zielstellung wird anhand eines Probandentests evaluiert. Dafür wurden verschiedene Erfolgskriterien festgelegt, welche einzelne Teilaspekte der Zielstellung validieren. Dabei wird zum einen das Empfinden der Testspieler evaluiert, insbesondere ob sie ihre Ideen und Strategien im Spiel frei ausleben konnten, und zum anderen ob das Tool für den  Spielleiter verständlich und intuitiv nutzbar ist. Eine genaue Beschreibung der Erfolgskriterien, sowie eine Quantifizierung findet sich in \ref{sec:Erfolgskriterien}

  
% Auskommentiert, damit's ein bisschen schöner aussieht bei der Abgabe ;)
\section{Gliederung}
\label{sec:Gliederung}
In dem \hyperref[background]{folgenden Kapitel} wird auf verschiedene Grundlagen eingegangen, welche zum Verständnis dieser Arbeit nötig sind. In \ref{concept} wird auf Basis dieser Grundlagen ein Konzept zur Beantwortung der wissenschaftlichen Frage ausgearbeitet und in \ref{implementation} die gewählte Implementierung erläutert.\newline \ref{evaluation} evaluiert mit definierter Erfolgskriterien die wissenschaftliche Frage über den Prototypen. In Kapitel 6 findet sich schließlich das gezogene Fazit und ein Ausblick auf weiter Arbeitspakete und Vertiefungsrichtungen.\newline Weitere Dokumente wie Bedienhandbuch und Fragebögen finden sich im Anhang.

